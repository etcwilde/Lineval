% vim:set et sw=2 ts=4 tw=72:

\section{Merge-Tree model}
\label{sec:mergetree}

The \mt model abstracts the DAG of the repository into a set of \mt{s}.
Each \mt is rooted at a merge into the master branch. The leaves of this
tree are the commits and the merges are the inner nodes. A \mt is built
recursively with every merge in the \mt merging a sub-\mt. Any merges
that a commit must pass through to reach the master branch become an
inner node of the tree. Effectively, a \mt is a tree that shows the path
that commits follow in their way to the master branch of a repository.
The \mt{s} also help understand how commits were grouped together in
order to be integrated. In this model, not only has the DAG been
inverted (and simplified), but the entire notion of parent-child
relationship has been reversed. Due to this property, the terminology
will be different depending on the model we are referring to; when we
are referring to the \mt, the parent is the next node toward the merge
into the master branch, or the root of the tree. When referring to the
DAG, the parent relationship is in the opposite direction, from the root
toward the branch-point. In addition to identifying the path that a
commit took to being merged, we are able to aggregate the commit
metadata at merges as we retain the parent-child and child-parent
relationship for each event and know which commits belong to the merge.

To illustrate this model we will use a small example: assume the commits
represented in Figure~\ref{fig:repoEvents} show the sequence of events
in a repository. The sequence starts with the initial commit in the
master branch of the master repository at time $t_0$. Repository event 1
is a commit, which gets forked into a separate repository, \textit{Repo
  A}, where another commit is made, event 2. Event 5 is a merge event,
merging event 2, 3, and 4 into \textit{Repo A}. Event 5 is branched
from, commit 6 happens in the new branch, while commit 7 is added
simultaneously to the original branch in \textit{Repo A}. Events 11 and
12 are both merge events, merging changes made in \textit{Repo A} into
the master branch of the master repository. As every repository is a
first-class repository, including local copies and forks, git does not
distinguish between forked repositories and branches, and in neither
case does it explicitly record where a commit was made. In this case,
commits are performed in various repositories and branches. The DAG
representation of these events is shown in Figure~\ref{fig:repoDAG}.

The DAG does not retain information about where a commit was originally
created. The \mt form (Figure~\ref{fig:repoTree}) does not completely
rebuild the lost information, but is able to re-create the sequence of
merges required to merge the event, and the depth, in the tree, of that
event. The \mt does not include information about how the repository
events that are the children of a merge are related. The order of the
children does not matter.

Combining the DAG and the \mt results in a hybrid structure
(Figure~\ref{fig:repoDAGTree}) that no only shows which merges a commit
passes through, it rebuilds the order of those commits. In the example
shown, this hybrid structure makes it clear that merge 9 merges the
changes made in 6 and 8 into the commit 7. Finally, 11 merges merge 9
into the master branch.

\begin{figure}[htbp]
  \centering
  \begin{tikzpicture}[auto, on grid, semithick, state/.style={circle, text=black}]
    \foreach \x in {0, 1, 2, 3, 4, 5, 6, 7}
    \draw[shift={(\x + 0.5, -0.5)}, color=black] (0cm, 4cm) -- (0pt, -0.2cm);

    \node[state, draw=chartblue] (1) {1};
    \node[state, draw=chartyellow, above right= of 1] (2) {2};
    \node[state, draw=chartmagenta, above right= 2cm and 1cm of 2] (3) {3};
    \node[state, draw=chartblue, right= 2cm of 1] (4) {4};
    \node[state, draw=chartyellow, above right=of 4] (5) {5};
    \node[state, draw=chartred, above right=of 5] (6) {6};
    \node[state, draw=chartyellow, right=of 5] (7) {7};
    \node[state, draw=chartmagenta, above right= 2cm and 1cm of 7](8) {8};
    \node[state, draw=chartyellow, right= 2cm of 7] (9) {9};
    \node[state, draw=chartyellow, right=of 9] (10) {10};
    \node[state, draw=chartblue, below right=of 9] (11) {11};
    \node[state, draw=chartblue, below right=of 10] (12) {12};

    \draw (12) edge[-stealth] (11) edge[chartyellow, -stealth] (10);
    \draw (11) edge[-stealth] (4) edge[chartyellow, -stealth] (9);
    \draw (10) edge[chartyellow, -stealth] (9);
    \draw (9) edge[chartmagenta, -stealth] (8) edge[chartred, -stealth] (6)
              edge[chartyellow, -stealth] (7);
    \draw (8) edge[chartmagenta, -stealth] (7);
    \draw (7) edge[chartyellow, -stealth] (5);
    \draw (6) edge[chartred, -stealth] (5);
    \draw (5) edge[chartmagenta, -stealth] (3) edge[chartyellow,-stealth] (2)
              edge[chartyellow, -stealth] (4);
    \draw (4) edge[-stealth] (1);
    \draw (3) edge[chartmagenta, -stealth] (2);
    \draw (2) edge[chartyellow, -stealth] (1);

    \node [draw=chartblue, below = 1.5cm of 1] (l1) {Master};
    \node [draw=chartyellow, right = 1.5cm of l1] (l2) {Repo A};
    \node [draw=chartred, right = 2.5cm of l2] (l3) {Branch of Repo A};
    \node [draw=chartmagenta, right= 2.5cm of l3] (l4) {Repo B};

        \foreach \x in {0, 1, 2, 3, 4, 5, 6, 7, 8}
    \node[shift={(\x, -0.6)}, color=black] {$t_\x$};
  \end{tikzpicture}
  \caption{An example sequence of events performed in different
    repositories. The horizontal axis represents time. The branches and
    repositories are aligned horizontally, and color-coded. Each commit
    points to its parent. The initial commit is at time $t_0$, and the
    head is at $t_8$.}
  \label{fig:repoEvents}
%\vspace{-3mm}
\end{figure}

\begin{figure}[htbp]
  \centering
  \begin{tikzpicture}[auto, on grid, semithick, state/.style={circle, text=black, black}]
    \node[state, black] (1) {1};
    \node[state, black, above right= of 1] (2) {2};
    \node[state, black, above right= 2cm and 1cm of 2] (3) {3};
    \node[state, black, right= 2cm of 1] (4) {4};
    \node[state, black, above right=of 4] (5) {5};
    \node[state, black, above right=of 5] (6) {6};
    \node[state, black, right=of 5] (7) {7};
    \node[state, black, above right= 2cm and 1cm of 7](8) {8};
    \node[state, black, right= 2cm of 7] (9) {9};
    \node[state, black, right=of 9] (10) {10};
    \node[state, black, below right=of 9] (11) {11};
    \node[state, black, below right=of 10] (12) {12};

    \draw (12) edge[-stealth] (11) edge[-stealth] (10);
    \draw (11) edge[-stealth] (4) edge[-stealth] (9);
    \draw (10) edge[-stealth] (9);
    \draw (9) edge[-stealth] (8) edge[-stealth] (6)
              edge[-stealth] (7);
    \draw (8) edge[-stealth] (7);
    \draw (7) edge[-stealth] (5);
    \draw (6) edge[-stealth] (5);
    \draw (5) edge[-stealth] (3) edge[-stealth] (2)
              edge[-stealth] (4);
    \draw (4) edge[-stealth] (1);
    \draw (3) edge[-stealth] (2);
    \draw (2) edge[-stealth] (1);
  \end{tikzpicture}
  \caption{DAG representation of the commits represented in
    Figure~\ref{fig:repoEvents}. The DAG loses information about which
    repository the commit is performed in and through which merges it
    has passed on its way to the master branch. The DAG does not even
    distinguish the master branch from other branches.}
  \label{fig:repoDAG}
%\vspace{-3mm}
\end{figure}

\begin{figure}[htpb]
  \begin{center}
    \begin{tikzpicture}[auto, on grid, semithick, node distance=1cm, state/.style={circle, text=black, minimum size=7mm}]

      \node[state, draw=chartblue] (1) {1};
      \node[state, draw=chartblue, right=of 1] (4) {4};
      \node[state, draw=chartblue, right=of 4] (11) {11};
      \node[state, draw=chartblue, right=2.5cm of 11] (12) {12};

      \node[state, draw=chartyellow, above right= 1cm and 0.5cm of 11] (7) {7};
      \node[state, draw=chartyellow, above left= 1cm and 0.5cm of 11] (5) {5};
      \node[state, draw=chartyellow, right=of 7] (9) {9};
      \node[state, draw=chartyellow, left=of 5] (2) {2};
      \node[state, draw=chartmagenta, above=of 5] (3) {3};
      \node[state, draw=chartmagenta, above left=1cm and 0.5cm of 9] (6) {6};
      \node[state, draw=chartmagenta, above right=1cm and 0.5cm of 9] (8){8};
      \node[state, draw=chartyellow, above=of 12] (10) {10};

      \draw (11) edge[chartyellow, stealth-] (2) edge[chartyellow, stealth-] (5) edge[chartyellow, stealth-] (7) edge[chartyellow, stealth-] (9)
            (5) edge[chartmagenta, stealth-] (3)
            (9) edge[chartmagenta, stealth-] (6) edge[chartmagenta, stealth-] (8)
            (12) edge[chartyellow, stealth-] (10);
    \end{tikzpicture}
  \end{center}
  \caption{The \mt{s} computed for each commit in
    Figure~\ref{fig:repoDAG} showing the path that each commit takes to
    be merged into the master branch of the repository. This does not
    indicate how the events being merged are related. We retained the
    numerical order of the events, but the order can be arbitrary.}
  \label{fig:repoTree}
\end{figure}

\begin{figure}[htbp]
  \centering
  \begin{tikzpicture}[auto, on grid, semithick, state/.style={circle, text=black}]

    \draw[chartblue]
      (-0.5, -0.5) -- (8.5, -0.5) -- (8.5, 0.5) -- (-0.5, 0.5) -- (-0.5, -0.5);

    \node[state, draw=chartblue] (1) {1};
    \node[state, draw=chartyellow, above right= of 1] (2) {2};
    \node[state, draw=chartmagenta, above right= 2cm and 1cm of 2] (3) {3};
    \node[state, draw=chartblue, right= 2cm of 1] (4) {4};
    \node[state, draw=chartyellow, above right=of 4] (5) {5};
    \node[state, draw=chartred, above right=of 5] (6) {6};
    \node[state, draw=chartyellow, right=of 5] (7) {7};
    \node[state, draw=chartmagenta, above right= 2cm and 1cm of 7](8) {8};
    \node[state, draw=chartyellow, right= 2cm of 7] (9) {9};
    \node[state, draw=chartyellow, right=of 9] (10) {10};
    \node[state, draw=chartblue, below right=of 9] (11) {11};
    \node[state, draw=chartblue, below right=of 10] (12) {12};

    \draw (2) edge[chartyellow, -stealth] (5)
          (3) edge[chartmagenta, -stealth](5)
          (5) edge[chartyellow, -stealth] (7)
          (7) edge[chartyellow, -stealth] (9)
          (6) edge[chartred, -stealth] (9)
          (8) edge[chartmagenta, -stealth] (9)
          (9) edge[chartyellow, -stealth] (11)
          (10) edge[chartyellow, -stealth] (12);


  \end{tikzpicture}
  \caption{The \mt applied to the inverted DAG to show the order that
    commits are created, as well as the merge sequence for the events in
  Figure~\ref{fig:repoEvents}. This requires a post-processing step
  involving the \mt produced by Algorithm~\ref{fig:alg} and the DAG.}
\label{fig:repoDAGTree}
\vspace{-3mm}
\end{figure}

\subsection{Computing the \mt of the DAG of Linux}

Computing the \mt from a DAG for any repository may not be possible;
however, certain features of the development process of Linux make it
feasible to compute the \mt for the Linux repository. First, the master
branch of Linux is maintained by Linus Torvalds, and only Linus has
write access to it. We have verified this assertion in previous
research~\cite{German2015}. We have developed a heuristic that is
presented in Algorithm~\ref{fig:alg}. In short, the algorithm first
identifies the commits made directly to the master branch, whereafter it
recursively determines the shortest path, using the DAG, from each
commit to the master branch using the inverted DAG.\@

\begin{algorithm}
        \caption{Computing the \mt of Linux from the DAG}\label{fig:alg}
        \begin{algorithmic}[1]
                \Function{ComputeMergeTree}{DAG}: tree
                % \State {\# Compute the tree from the DAG of Linux repository.}
                % \State {\# Returns $Tree$, a graph containing every commit }
                % \State {\# in DAG with the path it followed to master.}
                \State $head \gets \textit{Head of master of git repository}$
                \State $master \gets \textit{traverse DAG from head using }$
                \State \quad\quad\quad\quad $\textit{first parent until reaching root}$
                \State $nodes(Tree) \gets nodes(DAG)$
                \State \Function{MergeAtMaster}{cid}
                \State {\# Returns $(depth, merge, next)$}
                \State {\# Helper function}
                \State {\# Compute the closest merge into master, }
                \State {\# setting the children on the way to master.}
                \If {\textit{cid in master}}
                \State \Return $(0, cid, \varnothing)$
                \EndIf
                \State {$d \gets \infty$}
                \State {\# Traverse the inverted DAG}
                \For{$c \in children(cid, DAG)$}
                \State $(d_c, merge_c, next_c) \gets MergeAtMaster(c)$
                \If {$IsMerge(c)$}
                \State $fp \gets FindFirstParent(c)$
                \If {$fp \neq cid$}
                \State $d_c \gets d_c + 1$
                \State $next_c \gets c$
                \EndIf
                \EndIf
                \State {\# Find the shortest path}
                \If {$d_c < d$}
                \State $(d, m, next) \gets (d_c, merge_c, next_c)$
                \ElsIf{ $d_c = d$ }

                \State {\# Use the time as a tie-breaker}
                \If {$ cTime(merge_c) < cTime(m) $}
                \State $(m, next) \gets (merge_c, next_c)$
                \EndIf
                \EndIf
                \EndFor
                \State {\# $c$ is the commit that follows $cid$}
                \State {\# in its way to master}
                \State add edge $(cid, next)$ to $Tree$
                \State \Return $(d, m, next)$
                \EndFunction

                \State {\# Compute the distance for each commit}
                \State {\# discarding result}
                \For{$c \in nodes(DAG)$}
                \State $MergeAtMaster(c)$
                \EndFor
                \State \Return $Tree$
                \EndFunction
        \end{algorithmic}
\end{algorithm}

The algorithm is broken into two phases. The first is determining which
repository events are on the master branch. This is done by traversing
the first parent from the master branch reference to the commit that has
no parents. The second phase is encompassed by the function
$MergeAtMaster$ which determines, for each commit, which merge the
commit is merged at, the depth (as variable $d$ in the algorithm), and
the next merge on the path to the master branch. The function
$MergeAtMaster$ has two parts, the first for determining the depth, from
the master branch, that the repository event is at. The second phase
determines the merge into the master branch, and the next merge on the
way to the master branch. The distance is by shortest path, staying as
close to the master branch as possible. If there is a tie between two
paths, the path that merges into the master branch sooner is taken.

To demonstrate the behaviour of the algorithm, we use a short example,
computing the merge at commit 5 in Figure~\ref{fig:repoEvents}.
$MergeAtMaster$, recurses along the children of the nodes it visits.
Eventually every child of every node along the path will be visited at
least once. Without loss of generality, suppose that the path recursed
along is from node 5 to 6, 9, 10, and finally 12. 

The depth for each, except 12 (a merge into the master branch), is
initialized to infinity, the merge into master is blank, and the next
merge is blank. Merges into master trivially have a distance of 0 from
the master branch, and it merges itself into the master branch. The
recursion at 12 returns the triple $(0, \varnothing, 12)$ to the call
from 10. 12 is a merge commit and 10 is not the first parent, so the
temporary depth, $d_c$, is incremented to 1 and the temporary next
merge, $next_c$, is changed to 12. 1 is less than infinity, so the depth
is set to 1, the merge to 12, and the next to 12. This returns the
triple $(1, 12, 12)$ to the call from 9. 9 is the first parent of 10, so
no changes are made to the temporary variables.

The call to 9 recurses to the second child, 11. 11 is a merge into the
master so it returns $(0, \varnothing, 11)$ to the call from 9. 9 is not
the first parent of 11, so the $d_c$ is incremented to 1 and $next_c$ is
changed to 11. The distances $d_c$ and $d$ are the same, so to break the
tie, we use the time. 12 was merged after 11, so 11 replaces 12 as the
merge into the master branch for 9, as well as being the next merge. The
call for 9 returns the triple $(1, 11, 11)$ to the call for 6. 6 is not
the first parent of 9, so $d_c$ is incremented and $next_c$ is changed
to 9, as 9 merges 6. 2 is less than infinity, so the $d$ is changed to
2, the merge to 11, and the next merge to 9. The call to 6 returns the
triple $(2, 11, 9)$ to the call for 5. 

The call for 5 recurses on the second child of 5, calling on 7, which
calls 8, and then 9. 9 can continue, but if the implementation of the
algorithm uses memoization, the call to 9 can immediately return the
triple $(1, 11, 11)$ to the call for 8, and avoid an exponential
runtime. 8 is not the first parent of 8, so $d_c$ is incremented to 2
and $next_c$ is changed to 9. 2 is less than infinity, so $d$ is changed
to 2, $merge$ to 11, and $next$ to 9. The call to 8 returns the triple
$(2, 11, 9)$ to the call for 7, which recurses on the second child of 7,
9. 9 returns the triple $(1, 11, 11)$. 7 is the first parent of 9, so
the depth is not incremented. $d_c$ is less than $d$, so $d$ is changed
to 1, $m$ to 11, and $next$ to 11, returning $(1, 11, 11)$ to the call
for 5. $d_c$ is less than $d$, so $d$ is changed to 1, $m$ to 11, and
$next$ to 11. There are no other children, so the function halts.



\subsection{Evaluation}

Merges that do not have conflicts provide information to verify this
heuristic. If a merge does not contain a conflict, it records a summary
of the commits that it merges. See Figure~\ref{fig:sampleMerge} for an
example. This summary contains a list of the first 20 non-merge commits
in the merge, including their one-line log description, and the total
number of non-merge commits in the merge.

\begin{figure}[htbp]
        \centering
        \vspace{3mm}
        \fontsize{7}{9}
        \begin{Verbatim}[frame=single]
Merge: 8cbd84f fd8aa2c
Author: Linus Torvalds <torvalds@linux-foundation.org>
Date:   Tue Aug 10 15:38:19 2010 -0700

Merge branch 'for-linus' of git://neil.brown.name/md

* 'for-linus' of git://neil.brown.name/md: (24 commits)
md: clean up do_md_stop
[... edited for the sake of space]
md: split out md_rdev_init
md: be more careful setting MD_CHANGE_CLEAN
md/raid5: ensure we create a unique name for kmem_cache...
...
        \end{Verbatim}
        \caption{Example of how merges record a subset of commits being merged. The
                commit only shows the first 20 one-line summaries messages for the 24
                non-merge commits it merged. The ending ``\ldots'' is part of the log
                and represents that other commits were merged.}
        \label{fig:sampleMerge}
  \vspace{2mm}
\end{figure}

We used this information to evaluate the accuracy of the \mt model
extracted from the DAG\@. The method we followed started with the
extraction of the commit history up to July 20, 2016 form the Linux
repository. We computed the \mt of every commit until that date. Since
Linus Torvalds mostly does merging directly into the master branch, we
assumed that every merge by Linus is the root of a \mt, later detecting
merges that do not merge into the master branch and removing them from
the set of root merges. As described above, the log of a merge-commit
usually contains the number of commits in the merge the first 20
summaries of commits being merged. We extracted merges by Linus Torvalds
using the command \verb|git log --merges --author='Torvalds'| and
compared the number of commits stated in the log message with the number
of commits in the \mt. We also used the summaries
of the commits found in the merge (not necessarily all---see above) to
make sure those commits were in their corresponding \mt. For example, for
the merge in Figure~\ref{fig:sampleMerge} we would expect that the \mt
rooted at \mycode{8cbd84f} contains 24 commits, and the one-line
summaries corresponds to commits in that \mt. We also inspected those
with differences to make sure they were true errors. The results can be
summarized as follows:

\begin{itemize}

  \item

    Five merges were false-errors because their logs did not contain
    accurate information (were probably edited by hand). For example in
    \mycode{42a579a0f\ldots} one commit summary was missing (the line was
    empty), in \mycode {c55d267\ldots} the summaries were reordered.

  \item

    The heuristic correctly identified that 79 of Linus merges (between
    Jun 7, 2014 and Jun 2, 2014) were made to a branch (not master).
    This branch was merged at \mycode{3f17ea6d\ldots} which contained 6809
    commits.

  \item

    The heuristic worked perfectly until Sept 4, 2007, the earliest date
    that it could be verified.  Before this date, and until Dec 12,
    2006, merges did not include a summary of the commits they included,
    hence making it impossible to verify; during this period, however,
    we correctly identified the merges by Linus into the master branch.

  \item

    Before Dec. 12, 2006 (1542 merges) our heuristic breaks due to the
    presence of a \foxtrot commit (\mycode{c436688\ldots}),
    which confounded the true master branch.

\end{itemize}

In summary, of the merges after Sept. 4, 2007, our heuristic was correct
in 100\% of the 16,680 commits. It failed in 1,542 commits before Dec.
12, 2006 and in 836 it appears to be correct (Dec 7, 2006 to Sept 4,
2007).

%%% Local Variables:
%%% mode: latex
%%% TeX-master: "lineval.tex"
%%% End:
