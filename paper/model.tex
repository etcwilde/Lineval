% vim:set et sw=2 ts=4 tw=72:
The \mt model inverts the DAG, so that reference point from the parent
to the child instead of from child to parent. This provides a clear path
from a commit to the closest merge into the master branch. We define the
distance as a measure of time from when the commit was initially
commited to the date it was merged into the project. The merge into the
master branch becomes the root of the tree, and the commits are the
leaves. Any merges that that a commit must pass through to reach the
master branch become an inner node of the tree. The structure is able to
help a user identify the path that the commit took to reach the master
branch, as well as understand how commits were grouped together in order
to be integrated. In this model, not only has the DAG been inverted, but
the entire notion of parent-child relationship has been reversed. The
terminology surrounding the parent-child relationship will prove to be
confusing through the rest of the paper, when we are referring to the
\mt, the parent is the next node toward the merge into the master
branch, or the root of the tree. When referring to the DAG, the parent
relationship is in the opposite direction, from the root toward the
branch-point. In addition to identifying the path that a commit took to
being merged, we are able to aggregate the commit metadata at merges as
we retain the parent-child and child-parent relationship for each event.

A short example: assume the commits represented in
Figure~\ref{fig:repoEvents} show the sequence of events in a repository.
The sequence starts with the initiall commit in the master branch of the
master repository at time $t_0$. Repository event 1 is a commit, which
gets forked into a separate repository, \textit{Repo A}, where another
commit is made, event 2. Event 5 is a merge event, merging event 2, 3,
and 4 into \textit{Repo A}. Event 5 is branched from, event 6 happens in
the new branch, while event 7 is added simultaneously to the original
branch in \textit{Repo A}. Events 11 and 12 are both merge events,
merging changes made in \textit{Repo A} into the master branch of the
master repository.

As every repository is a first-class repository, including local copies
and forks, Git does not distinguish between forked repositories and
branches, and in neither case does it explicitly record where a commit
was made. In this case, commits are performed in various repositories
and branches. The DAG representation of these events is shown in
Figure~\ref{fig:repoDAG}.

 Notice that the DAG loses information about
the master branch and the repository that the master branch is part of.
The merge-tree view of this DAG is visible in Figure~\ref{fig:repoTree}.
Note that the direction of the edges of the DAG have been inverted,
instead of pointing from the child to the parent, it points from the
parent to its children, forming a path to the master branch. Also
note that the DAG has been simplified, showing only a single edge on the
path to master for any commit.

\begin{figure}[htbp]
  \centering
  \begin{tikzpicture}[auto, on grid, semithick, state/.style={circle, text=black}]
    \foreach \x in {0, 1, 2, 3, 4, 5, 6, 7}
    \draw[shift={(\x + 0.5, -0.5)}, color=black] (0cm, 4cm) -- (0pt, -0.2cm);

    \node[state, draw=chartblue] (1) {1};
    \node[state, draw=chartyellow, above right= of 1] (2) {2};
    \node[state, draw=chartmagenta, above right= 2cm and 1cm of 2] (3) {3};
    \node[state, draw=chartblue, right= 2cm of 1] (4) {4};
    \node[state, draw=chartyellow, above right=of 4] (5) {5};
    \node[state, draw=chartred, above right=of 5] (6) {6};
    \node[state, draw=chartyellow, right=of 5] (7) {7};
    \node[state, draw=chartmagenta, above right= 2cm and 1cm of 7](8) {8};
    \node[state, draw=chartyellow, right= 2cm of 7] (9) {9};
    \node[state, draw=chartyellow, right=of 9] (10) {10};
    \node[state, draw=chartblue, below right=of 9] (11) {11};
    \node[state, draw=chartblue, below right=of 10] (12) {12};

    \draw (12) edge[-stealth] (11) edge[chartyellow, -stealth] (10);
    \draw (11) edge[-stealth] (4) edge[chartyellow, -stealth] (9);
    \draw (10) edge[chartyellow, -stealth] (9);
    \draw (9) edge[chartmagenta, -stealth] (8) edge[chartred, -stealth] (6)
              edge[chartyellow, -stealth] (7);
    \draw (8) edge[chartmagenta, -stealth] (7);
    \draw (7) edge[chartyellow, -stealth] (5);
    \draw (6) edge[chartred, -stealth] (5);
    \draw (5) edge[chartmagenta, -stealth] (3) edge[chartyellow,-stealth] (2)
              edge[chartyellow, -stealth] (4);
    \draw (4) edge[-stealth] (1);
    \draw (3) edge[chartmagenta, -stealth] (2);
    \draw (2) edge[chartyellow, -stealth] (1);

    \node [draw=chartblue, below = 1.5cm of 1] (l1) {Master};
    \node [draw=chartyellow, right = 1.5cm of l1] (l2) {Repo A};
    \node [draw=chartred, right = 2.5cm of l2] (l3) {Branch of Repo A};
    \node [draw=chartmagenta, right= 2.5cm of l3] (l4) {Repo B};

        \foreach \x in {0, 1, 2, 3, 4, 5, 6, 7, 8}
    \node[shift={(\x, -0.6)}, color=black] {$t_\x$};
  \end{tikzpicture}
  \caption{Example of a sequence of events performed in different
    repositories. The horizontal axis represents time. Each horizontal
    section represents a different branch and/or repository. Each commit
    points to its parent. The intial commit is at time $t_0$, and the
    head is at $t_8$.}
  \label{fig:repoEvents}
%\vspace{-3mm}
\end{figure}

\begin{figure}[htbp]
  \centering
  \begin{tikzpicture}[auto, on grid, semithick, state/.style={circle, text=black}]
    \node[state, draw=chartblue] (1) {1};
    \node[state, draw=chartyellow, above right= of 1] (2) {2};
    \node[state, draw=chartmagenta, above right= 2cm and 1cm of 2] (3) {3};
    \node[state, draw=chartblue, right= 2cm of 1] (4) {4};
    \node[state, draw=chartyellow, above right=of 4] (5) {5};
    \node[state, draw=chartred, above right=of 5] (6) {6};
    \node[state, draw=chartyellow, right=of 5] (7) {7};
    \node[state, draw=chartmagenta, above right= 2cm and 1cm of 7](8) {8};
    \node[state, draw=chartyellow, right= 2cm of 7] (9) {9};
    \node[state, draw=chartyellow, right=of 9] (10) {10};
    \node[state, draw=chartblue, below right=of 9] (11) {11};
    \node[state, draw=chartblue, below right=of 10] (12) {12};

    \draw (12) edge[-stealth] (11) edge[chartyellow, -stealth] (10);
    \draw (11) edge[-stealth] (4) edge[chartyellow, -stealth] (9);
    \draw (10) edge[chartyellow, -stealth] (9);
    \draw (9) edge[chartmagenta, -stealth] (8) edge[chartred, -stealth] (6)
              edge[chartyellow, -stealth] (7);
    \draw (8) edge[chartmagenta, -stealth] (7);
    \draw (7) edge[chartyellow, -stealth] (5);
    \draw (6) edge[chartred, -stealth] (5);
    \draw (5) edge[chartmagenta, -stealth] (3) edge[chartyellow,-stealth] (2)
              edge[chartyellow, -stealth] (4);
    \draw (4) edge[-stealth] (1);
    \draw (3) edge[chartmagenta, -stealth] (2);
    \draw (2) edge[chartyellow, -stealth] (1);
  \end{tikzpicture}
  \caption{DAG representation of the commits represented in
    Figure~\ref{fig:repoEvents}. The DAG loses information about which
    repository the commit is performed in and through which merges it
    has passed on its way to the master branch. The DAG does not even
    distinguish the master branch from other branches.}
  \label{fig:repoDAG}
%\vspace{-3mm}
\end{figure}

\begin{figure}[htbp]
  \centering
  \begin{tikzpicture}[auto, on grid, semithick, state/.style={circle, text=black}]

    \draw[chartblue]
      (-0.5, -0.5) -- (8.5, -0.5) -- (8.5, 0.5) -- (-0.5, 0.5) -- (-0.5, -0.5);

    \node[state, draw=chartblue] (1) {1};
    \node[state, draw=chartyellow, above right= of 1] (2) {2};
    \node[state, draw=chartmagenta, above right= 2cm and 1cm of 2] (3) {3};
    \node[state, draw=chartblue, right= 2cm of 1] (4) {4};
    \node[state, draw=chartyellow, above right=of 4] (5) {5};
    \node[state, draw=chartred, above right=of 5] (6) {6};
    \node[state, draw=chartyellow, right=of 5] (7) {7};
    \node[state, draw=chartmagenta, above right= 2cm and 1cm of 7](8) {8};
    \node[state, draw=chartyellow, right= 2cm of 7] (9) {9};
    \node[state, draw=chartyellow, right=of 9] (10) {10};
    \node[state, draw=chartblue, below right=of 9] (11) {11};
    \node[state, draw=chartblue, below right=of 10] (12) {12};

    \draw (2) edge[chartyellow, -stealth] (5)
          (3) edge[chartmagenta, -stealth](5)
          (5) edge[chartyellow, -stealth] (7)
          (7) edge[chartyellow, -stealth] (9)
          (6) edge[chartred, -stealth] (9)
          (8) edge[chartmagenta, -stealth] (9)
          (9) edge[chartyellow, -stealth] (11)
          (10) edge[chartyellow, -stealth] (12);


  \end{tikzpicture}
  \caption{Merge-tree view of the commits  represented in
    Figure~\ref{fig:repoEvents} showing the path they followed to reach
    the master branch. In this model the successors of each commit
    represents the path followed by that commit to reach the master
    branch.}
  \label{fig:repoTree}
\vspace{-3mm}
\end{figure}


\subsection{Computing the merge-tree of the DAG of Linux}

Computing the merge-tree from a DAG for any repository may not be
possible; however, certain features of the development process of Linux
make it feasible to compute the merge-tree for the Linux repository.
First, the master branch of Linux is maintained by Linus Torvalds, and
only Linus has write access to it. We have verified this assertion in
previous research~\cite{German2015}. We have developed a heuristic that
is presented in Algorithm~\ref{fig:alg}. In short, the algorithm first
identifies the commits made directly to the master branch; whereafter it
recursively determines the shortest path (in terms of time), using the
DAG, from each commit to the master branch using the inverted DAG.

\begin{algorithm}
        \caption{Computing the merge-tree of Linux Git's DAG}\label{fig:alg}
        \begin{algorithmic}
                \Function{ComputeMergeTree}{DAG}: tree
                \State {\# Compute the tree from the DAG of Linux repository.}
                \State {\# Returns $Tree$, a graph containing every commit }
                \State {\# in DAG with the path it followed to master.}

                \State $head \gets \textit{Head of master of git repository}$
                \State $master \gets \textit{traverse DAG from head using }$
                \State \quad\quad\quad\quad $\textit{first ancestor until reaching root}$
                \State $nodes(Tree) \gets nodes(DAG)$
                \State \Function{distance2Master}{cid} : seconds
                \State {\# Helper function}
                \State {\# Recursively compute shortest distance to master}
                \State {\# setting cid's successor (next) in its way to master.}
                \State {\# This function should be memoized. Otherwise it}
                \State {\# would run in exponential time.}
                \If {\textit{cid in master}}
                \State \Return 0
                \EndIf
                \State    $d \gets 	\infty$
                \State {\# Traverse the inverted DAG}
                \For{$c \in children(cid, DAG)$}
                \If {$c \in master$}
                \State $d_1 \gets commitTime(c)-commitTime(cid)$
                \Else
                \State {$d_1 \gets distance2Master(c)$}
                \EndIf
                \If {$d_1 < d $}
                \State $next \gets c$
                \State  $d \gets d_1$
                \EndIf
                \EndFor
                \State {\# $c$ is the commit that follows $cid$}
                \State {\# in its way to master}
                \State add edge $(cid, next)$ to $Tree$
                \State \Return $d$
                \EndFunction

                \State {\# Compute the distance for each commit}
                \State {\# discarding result}
                \For{$c \in nodes(DAG)$}
                \State $distance2Master(c)$
                \EndFor
                \State \Return $Tree$
                \EndFunction
        \end{algorithmic}
\end{algorithm}


\subsection{Evaluation}

Merges that do not have conflicts provide information to verify this
heuristic. If a merge does not contain a conflict, it records a summary
of the commits that it merges. See Figure~\ref{fig:sampleMerge} for an
example. This summary contains a list of the first 20 non-merge commits
in the merge, including their one-line log description, the full logs of
the merge commits that merge this subset, and the total number of
non-merge commits in the merge.

\begin{figure}[htbp]
        \centering
        {\fontsize{7}{9}
        \begin{verbatim}
Merge: 8cbd84f fd8aa2c
Author: Linus Torvalds <torvalds@linux-foundation.org>
Date:   Tue Aug 10 15:38:19 2010 -0700

Merge branch 'for-linus' of git://neil.brown.name/md

* 'for-linus' of git://neil.brown.name/md: (24 commits)
md: clean up do_md_stop
[... edited for the sake of space]
md: split out md_rdev_init
md: be more careful setting MD_CHANGE_CLEAN
md/raid5: ensure we create a unique name for kmem_cache...
...
        \end{verbatim}}\vspace{-5mm}
        \caption{Example of how merges record a subset of commits being merged. The
                commit only shows the first 20 one-line summaries messages for the 24
                non-merge commits it merged. The ending ``\ldots'' is part of the log
                and represents that other commits were merged.}
        \label{fig:sampleMerge}
\end{figure}

We used this information to evaluate the accuracy of the merge-tree
model extracted from the DAG\@. The method we followed started with the
extraction of the commit history up to July 20, 2016. We computed the
merge tree of every commit until then. Since Linus Torvalds mostly does
merging directly into master, we assumed that every merge by him is the
root of a merge-tree. As described above, the log of a merge-commit
usually contains the number of commits in the merge the first 20
summaries of commits being merged. We extracted merges by Linus Torvalds
using the command \mycode{log --merges --author='Torvalds} and compared
the number of commits according to the log with the number of commits in
the  merge-tree rooted in this commit. We also used the summaries of the
commits found in the merge (not necessarily all---see above) to make
sure those commits were in their corresponding merge tree. For example,
for the merge in Figure~\ref{fig:sampleMerge} we would expect that the
merge tree rooted at \mycode{8cbd84f} contains 24 commits, and the
one-line summaries corresponds to commits in that merge-tree. We also
inspected those with differences to make sure they were true errors.
The results can be summarized as follows:

\begin{itemize}

  \item

    Five merges were false-errors because their logs did not contain
    accurate information (were probably edited by hand). For example in
    \mycode{42a579a0f\ldots} one commit summary was missing (the line was
    empty), in \mycode {c55d267\ldots} the summaries were reordered.

  \item

    The heuristic correctly identified that 79 of Linus merges (between
    Jun 7, 2014 and Jun 2, 2014) were made to a branch (not master).
    This branch was merged at \mycode{3f17ea6d\ldots} which contained 6809
    commits.

  \item

    The heuristic worked perfectly until Sept 4, 2007, the earliest date
    that it could be verified.  Before this date, and until Dec 12,
    2006, merges did not include a summary of the commits they included,
    hence making it impossible to verify; during this period, however,
    we correctly identified the merges by Linus into master.

  \item

    Before Dec. 12, 2006 (1542 merges) our heuristic breaks due to the
    presence of a \foxtrot commit (\mycode{c436688\ldots}),
    which confounded the true master branch of a repository.

\end{itemize}

In summary, of the merges after Sept. 4, 2007, our heuristic was correct
in 100\% of the 16,680 commits. It failed in 1,542 commits before Dec.
12, 2006 and in 836 it appears to be correct (Dec 7, 2006 to Sept 4,
2007).
