We conducted a user study of \evan{NUMBER} participants to determine if
there is a statistically significant difference among users in
performing certain tasks on repositories between \textit{Linvis} and
either gitk or the command-line git interface. We were looking for
improvements in accuracy, time performance, and overall enjoyment among
users when performing summarization and intra-tree navigation, as these
are the tasks that merge-trees are design to simplify.

\evan{still need to figure out who our intended user is... ``git user''
  is a bit too broad.}

% TODO: Determine who the intended user is. "generic git user" is kind of broad.
% TODO: Add a note stating that we will refer to the primary branch as being the master branch

\subsubsection{Methodology}
\label{ssub:methodology}

\subsubsection{Commit Selection}
\label{ssub:commit_selection}

We selected two commits for the participant to draw and summarize by
first determining the size of tree in the first and second quartiles. We
found that between Linux 3.1 and 3.16, the size of the trees in the
first quartile only contained a single node, and that the size of the
trees in the second quartile resulted in trees up to seven nodes. The
third quartile contained trees up to 51 nodes, and the fourth quartile
contained one tree of 7217 nodes. It is worth mentioning that the tree
sizes in the fourth quartile increase very quickly, as the next largest
tree only contains 4708 nodes, and the third largest tree contains 2349
commits. We see in figure~\ref{fig:tree_size} how the number of trees
decreases as the size of the tree increases.

\begin{figure}[htpb]
\begin{center}
\caption{Number of merge tree occurrences over size of the merge tree}
\label{fig:tree_size}
\begin{tikzpicture}
  \begin{axis} [
    ylabel=Occurences,
    xlabel=Tree Size,
    grid=both,
    minor x tick num = 1,
    minor y tick num = 1,
    ymode=log
    ]
    \addplot[chartred] table[col sep=comma, x index={1}, y index={0}]{data/tree_size.csv};
  \end{axis}
\end{tikzpicture}
\end{center}
\end{figure}

% TODO: Finish the paragraph on complexity
The complexity of the merge trees varies greatly.

The complexity of these trees varies greatly as well, . % TODO: Finish this sentence



We decided to work with one tree of one node, and one tree of seven
nodes.

\subsubsection{Questions}
\label{ssub:questions}

We chose questions that were designed to test the two primary goals of
merge trees, compared with the results from gitk or command line git. 

\subsubsection{Results}
\label{ssub:results}

\subsubsection{Discussion}
\label{ssub:discussion}

