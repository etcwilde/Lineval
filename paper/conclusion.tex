\section{Conclusion}
\label{sec:conclusion}

In this paper, we describe a tree-based model, \mt, designed to provide git
users with a better explanation of the events occurring in a git
repository. The \mt model is rooted at the merge into the master branch
of a repository, where the leaves are the individual commits.

We implement a tool, \tool, using the \mt model, which allows a user to
filter the commits with a simple search interface. Once the desired
commit is found, \tool provides tabs for simple aggregated tables
showing the files that were modified, the authors that made the
modifications, among other information. \tool also provides three
visualizations, list tree, pack tree, and a \rt tree view. The list tree
enables easy searching for textual features, while the pack tree is good
for visualizing the structure of large \mt{s}, and the \rt tree is
better at visualizing the structure of small \mt{s}.

To ensure that our model and implementation are useful, we evaluated the
tool with a three-part user study. We tested the conceptual understanding
using the Gitk, ability to summarize merges into the master branch, and
ask for the opinions of the participants. We found that the participants
were unable to determine the commits and merges that are related to the
commits we tested with and build a conceptual understanding from the DAG
visualization. The correctness, accuracy, and timing of summarization
tasks were improved using \tool compared to Gitk. The participants
generally enjoyed the clean summarizations and simple visualizations
provided by the \mt-based visualizations in \tool.

Given that visualization of the model provides users with a better
understanding of events in the repository, we describe some issues with
identifying the master branch of the repository and a more generalized
approach to constructing \mt{s}. We implemented the generalized approach
as a Bitbucket plugin to verify that it is able to produce
visualizations for arbitrary bitbucket repositories. We found that it
was able to produce visualizations for many repositories, but was unable
to produce the visualizations of others due to rate-limiting. An
interesting side-effect of pruning the DAG is that some \mt{s} in the
repository can be visualized, while other trees are too large to
download the commit information for. The original algorithm design would
be unable to produce a visualization for any of the \mt{s}.

\mt{s} are a novel means of building a model which can be visualized and
summarized in an effective way. Participants in our study found
visualizations of the \mt to be more enjoyable for summarization tasks
than the visualizations of the DAG\@. While there are some
implementation issues with \tool, \mt visualizations are an effective
means of providing a clear conceptual understanding and simple
summarization of the events being merged into the master branch of a
repository.

%%% Local Variables:
%%% mode: latex
%%% TeX-master: "lineval.tex"
%%% End:
