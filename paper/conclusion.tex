\section{Conclusion}
\label{sec:conclusion}

\dmg{in this paper we describe, not develop}

In this paper, we develop a tree-based model, \mt, designed to provide git
users with a better explanation of the events occurring in a git
repository. The \mt model is rooted at the merge into the master branch
of a repository, where the leaves are the individual commits.

We implement a tool, \tool, using the \mt model, which allows a user to
filter the commits with a simple search interface. Once the user finds \dmg{user => they or he/she or users => they, tense issue?}
the commit that they are interested in, they are provided simple
aggregated tables showing the files that were modified, the authors that
made the modifications, among other information. \dmg{they is very ambiguous: are we talking authors, users, tables,
  files? be precise}
They are also provided
with three visualizations, a list tree, a pack tree, and a
Reingold-Tilford tree view. The list tree enables easy searching for
textual features, while the pack tree is good for visualizing the
structure of large \mt{s}, and the Reingold-Tilford tree is better
at visualizing the structure of small \mt{s}.

\dmg{not necessary: useful, and not only the model, but the implementation}
To ensure that our model is necessary, we evaluate the ability of users
to perform simple conceptual tasks \dmg{using linvis and comparing it with ... be precise}
, finding how a commit was merged into
the master branch of the Linux repository, and determining other commits
that were merged with the given merge. Finding that the participants in
our study were unable to perform this task \dmg{using gitk?, be precise, but you didn't ask them to compare based on
  their unability to perform the task, they were ask to do it nonetheless}, we compared the ability of
the participants to summarize key pieces of information aggregated in
these groups of commits. We asked for information about what files were
modified, the authors making the changes, among other tasks\dmg{among other tasks? tasks are T1...Tn. Don't overload
  uses of a word}. Finally, we
asked the participants about their preference in tools, and which
aspects of each they preferred. We found that the participants in our
study were able to identify the correct answer more quickly and
accuractly using the \mt visualization than they were able to with
visualizations of the DAG.

Given that visualization of the model provides users with a better
understanding of events in the repository, we describe some issues with
identifying the master branch of the repository and a more generalized
approach to constructing \mt{s}. We implement \dmg{this should be past tense} the generalized approach
as a Bitbucket plugin to verify that it is able to quickly produce
results online, unlike in the implementation of \tool.\dmg{the most important aspect is nto that it can quickly produce
  results, but that it can be applied to any repo in bitbucket}

Participants in our study found them \dmg{who/what is them?} to be more enjoyable for
summarization tasks than visualizations of the DAG.\@ While there are
some issues \dmg{with what tool? linvis?, \mt should be in the first phrase, not the ending phrase} with identifying the master branch in a robust manner, \mt
visualizations are an effective means of providing a clear conceptual
understanding and simple summarization of the events being merged into a
the master branch of a repository.


%%% Local Variables:
%%% mode: latex
%%% TeX-master: "lineval.tex"
%%% End:
