In this paper, we develop a tree-based model designed to provide git
users with a better explanation of the events occurring in a git
repository. The \mt model is rooted at the merge into the master branch
of a repository, where the leaves are the individual commits.

We implement a tool, \tool, which allows a user to filter the commits
with a simple search interface. Once the user finds the commit that they
are interested in, they are provided simple aggregated tables showing
the files that were modified, the authors that made the modifications,
among other information. They are also provided with three
visualizations, a list tree, a pack tree, and a Reingold-Tilford tree
view. The list tree enables easy searching for textual features, while
the pack tree is good for visualizing the structure of large merge
trees, and the Reingold-Tilford tree is better at visualizing the
structure of small merge trees.

To ensure that our model is necessary, we evaluate the ability of users
to perform simple conceptual tasks, finding how a commit was merged into
the master branch of the Linux repository, and determining other commits
that were merged with the given merge. Finding that the participants in
our study were unable to perform this task, we compared the ability of
the participants to summarize key pieces of information aggregated in
these groups of commits. We asked for information about what files were
modified, the authors making the changes, among other tasks. Finally, we
asked the participants about their preference in tools, and which
aspects of each they preferred. We found that the participants in our
study were able to correctly identify the correct answer more quickly
using the \mt model than they were able to with the DAG.

Given that the model provides users with a better understanding of
events in the repository, we describe some issues with identifying the
master branch of the repository and a more generalized approach to
constructing merge trees. We implement the generalized approach as a
bitbucket plugin to verify that it is able to quickly produce results
online, unlike in the implementation of \tool.



% TODO: You can tell that I am tired. This conclusion will need to be re-worked.

