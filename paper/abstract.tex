Natively, git uses a directed acyclic grph mode (DAG) with references pointing from
the (TODO: most recent commit?, it's not the HEAD because that moves around) toward
the initial commit in the repository. This model is useful and necessary to the
internal functionality of the version control system, but does not provide users
with a clear nderstand of the events occuring in the repository. Due to the
direction of the link in the DAG, visualziation tools, like gitk, are unable to
provide summarized aggregation of the commits at a given merge. Furthermore, the
visual DAG does not provide a clear explanation of the events taking place in the
repository.

Merge-Trees are a data structure derived from the DAG, that could potentially
alleiveate the issues with visualization built directly from the DAG. The trees are
defined such that the root is the merge into the master branch, and the leaves are
the individual commits. The references point in both directed, from each node to the
parent, and from the parent to the children. Depending on the complexity of the
merge, the path from leaf to root may pass through multiple internal nodes.

In this paper, we verify that the Merge-tree model is able to simplify the
construction of a conceptual image of the parts of a repository, and simiplify the
summarization tasks through a user study. We will continue by providing a more
generalized approach to constructing merge-tress, which will prune components of the
DAG that are not necessary.

% \evan{Change this when we get the actual results, this is from the pilot study} We
% find that although the DAG view used by Gitk provides more information about the
% repository, in many cases it povides too much information and overwhelms the user.

% \evan{Move this}
% In some cases, where the structure of the merge tree is flat, that is to say that it
% has no inner merges, users are able to easily determine which commits are merged
% together, but are unable to determine which merge is the final merge is the root of
% the tree.
