% vim:set et sw=2 ts=4 tw=72:
With an average of more than 900 top-level merges into the Linux kernel
per release, many containing hundreds of commits and some containing
thousands, maintenance of older versions of the kernel becomes nearly
impossible.  Various commercial products, such as the Android platform,
run older versions of the kernel. Due to security, performance, and
changing hardware needs, maintainers must understand what changes
(commits) are added to the current version of the kernel since the last
time they inspected it in order to make the necessary patches.

Current tools provide information about repositories through the
directed acyclic graph (DAG) of the repository, which is helpful for
smaller projects. However, with the scale and number of branches in the
kernel the DAG becomes overwhelming very quickly. Furthermore, the DAG
contains every ancestor of every commit, while maintainers are more
interested in how and when a commit arrives to the official Linux
repository.

In this paper, we propose the merge-tree, a simplified transformation of
the DAG of the Linux git repository that shows the way in which commits
are merged into the master branch of Linux. Using the merge-tree, we
build \tool, a tool that is designed to allow users to explore how
commits are merged into the Linux kernel.
