% vim:set et sw=2 ts=4 tw=72:
% Jun 24, 2017
\subsection{Threats to Validity}
\label{sub:threats_to_validity}

While git comes bundled with gitk and the \verb|git log --graph|
command, many of the users in our study were unfamiliar with the DAG
shown by these tools. A potential reason for this may be that unless a
repository reaches sufficient complexity, both in branching and number
of commits, these tools are not necessary for the normal functionality
of git. However, while they were unfamiliar with either merge trees or
the DAG, these users were still able to better understand and summarize
the structure of the repository using merge trees.

Each question was evaluated independently of the results from the
previous question in the summarization tasks. Had the correct answers
been adjusted to match the participants' understanding of which commits
were involved in a merge tree, the results of the summarization would
have likely been very different. Another approach would to have provided
the participants with the correct tree once they had completed the
conceptual tasks, and the merge task set in the summarization task set.


\subsection{Future Work}
\label{sub:future_work}

% Inter Navigation
% TODO: verify this with the study

While the merge-tree model provides users with a better conceptual image
of what is occuring within a single merge tree, we have not exprimented
or tested the potential gains of navigation between trees. At this time,
the Linvis search provides results grouped by merge-tree root instead of
as a single block of results. The results are generally similar, with
each merge tree being merged at a different point in time, or to a
different version of the kernel. A future extension of this may
incorporate some information about how a given cluster changes over
time. To do this, a technique would need to be developed to cluster
merge-trees into modules, and then ordering the merge-trees based on
time or release.

% TODO: ensure that linvis has been referenced/defined

Perform a user study over multiple repositories using different merge
structures to see how quickly new members of a team can understand the
merge procedure. % Yeah, this is a big one, actually.


