% vim:set et sw=2 ts=4 tw=72:

We constructed a web-based tool called \tool in order to navigate and
inspect the \mt model of the kernel. Creating the web-based tool enables
users to use the system without having to install additional software or
store a large database, which makes it more accessible, easily
maintainable, and platform independent.

To navigate and inspect the merge-tree view of the kernel we created a
web-based tool called \tool. Creating a web-based tool enables users to
use the system without having to install additional software or store a
large database, making it more accessible, more easily maintainable, and
platform independent. \tool uses the following mechanisms to reach our
goals of better navigation and better explanation of the selected
changes.

\begin{itemize}
        \item Filter by searching
        \item View aggregated summaries of authors, files, and modules
        \item Visualize \mt{s}
\end{itemize}

\subsection{Search}

\tool provides a search engine for navigating within the kernel,
filtering commits that are not relevant. The search engine takes a
plain text query from the user and returns the results that are relevant
in the order of relevancy. When computing relevancy, the engine uses the
commit log, the author name, the filenames, the commit hash, the date
the commit was authored, and the date the commit was committed.

\begin{figure}[htpb]
  \centering
  \includegraphics[width=\linewidth]{figures/linvis/search_results.png}
  \caption{A single merge tree in the search results of \tool, showing
    the link to the root at the top and a table of the relevant commits
    in the bottom, sorted by relevance.}
  \label{fig:search_results}
\end{figure}

Before presentation, the results are grouped by merge tree. Each
merge tree group has a link to the root-node of the tree at the top,
followed by a table of commits and merges within the tree that match the
query, shown in Figure~\ref{fig:search_results}. The table includes the
relevancy rank of the entry, the commit log message preview, author, the
date the repository event was committed, and the date it was authored.
The merge tree groups are ordered based on the mean of the relevancy
scores of the relevant results in that tree.

\subsection{Summarization}
\label{sub:summarization}

\tool uses seven tabs to show the information and visualizations for a
selected commit or merge. The message tab shows the full commit log
message. This does not include the diff, but given the commit hash, this
information can be found directly from the repository. The files tab
shows an aggregated table of all files that were modified in a merge. It
includes metrics like the number of lines added, lines removed, total
lines modified, and the delta, summed across all commits in the merge
tree that modify this file. A details drop-down button allows a user to
see exactly which commit makes the changes, as shown in
Figure~\ref{fig:linvis_files}. If the current repository event being
viewed is a commit, the aggregate views will only show modifications
made by the commit. The modules tab shows the modules modified in the
merge tree. Like the files tab, the modules tab uses a table to show the
name of the module, the number of commits that are in the merge tree
that work with the module, and a details button to provide the links to
those commits.

\begin{figure}[htpb]
  \centering
  \includegraphics[width=\linewidth]{figures/linvis/linvis_files.png}
  \caption{Table showing the modified files in a merge, with the second
  entry expanded to show the commit that makes the changes.}
  \label{fig:linvis_files}
\end{figure}

The authorship tab is very similar to the files tab, but shows the
authorship information. It shows the sum of the number of lines added,
removed, modified, and the delta within the \mt. It also shows
the number of files that were modified by the author. The details tab is
organized slightly differently. Instead of organizing these results by
commit, the details are organized by file, showing which files were
modified by the author in this \mt (Figure~\ref{fig:linvis_authors}).

\begin{figure}[htpb]
  \centering
  \includegraphics[width=\linewidth]{figures/linvis/linvis_authors.png}
  \caption{Table showing the authors who made changes in a merge. The
    entry for Randy Dunlap is expanded, showing the modifications to
    each file that were made by Randy.}
  \label{fig:linvis_authors}
\end{figure}


\subsection{Visualization}
\label{sub:visualization}

The ability to easily visualize the integration of commits into a
project is what makes \tool unique from other tools. \tool implements
three visualizations, the list tree, pack tree, and \rt tree.

\subsubsection{List Tree}

The list tree, depicted in Figure~\ref{fig:linvis_list_tree}, is
constructed as a series of nested lists. The nesting indicates the
parent-child relationship. This visualization is text-based enabling
fast navigation to commits using the built-in search fin most
web-browsers. The list tree is rooted at the current repository event;
if the current event being inspected is a commit, only that commit will
be shown. Conversely, if the selected event is the root, the entire
merge tree will be shown.

\begin{figure}
        \centering
        \includegraphics[width=0.9\linewidth]{figures/linvis/linvis_list_tree.png}
        \caption{The list tree visualization}
        \label{fig:linvis_list_tree}
\end{figure}

\subsubsection{\rt Tree}

The \rt tree\cite{Reingold1981}, depicted for two trees in
Figure~\ref{fig:study_commits}, is the classic tree visualization, with
the root at the top, leaves at the bottom, and edges between them
showing the parent-child relationship. This tree visualization works
very well in many cases, but does not easily visualize very wide trees,
with many repository events on a single level.

When a user first looks at the visualization, the current repository
event is placed in the center. Furthermore, it is highlighted in bright
orange. Commits, the leaves, are shown in white, while the merges are
colored shades of blue where lighter blue indicates fewer child events
of that node, and darker blue indicates more children.

\subsubsection{Pack Tree}

Pack trees\cite{Wang2006} are useful for displaying an overview of large
data sets. The initial goals for the pack tree were for visualizing file
systems, which are similar to git repositories in that they are
relatively shallow, but very wide trees. The tree is represented by sets
of nested circles. The largest circle, containing all of the other
nodes, is the root, while the smallest circles are the leaves. In our
representation, the root maps to the merge into the master branch, while
the leaves are the commits.


\begin{figure}[htpb]
  \centering
  \includegraphics[width=0.8\linewidth]{figures/linvis/linvis_bubble.pdf}
  \caption{The pack tree visualization, with the root depicted as the
    outer-most circle, containing all other nodes, the leaves depicted
    as white circles containing no other nodes. The currently selected
    event is shown in orange.}
  \label{fig:linvis_pack}
\end{figure}

We depict the pack tree in Figure~\ref{fig:linvis_pack}. Like with the
\rt tree, \tool uses white to indicate the individual commits, and
orange to indicate the currently selected node. The inner merges are
colored with shades of blue to indicate the depth in the tree. Darker
shades of blue indicate that the merge is deeper in the tree.
