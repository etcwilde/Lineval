% vim:set et sw=2 ts=4 tw=72:
% Jul 04, 2017
% Evan Wilde

\section{Related Works}
\label{sec:related_works}

Version control systems monitor the development lifetime of software
projects. This makes the version control system vital in providing
information about how a software project is being developed. To do this,
the user must be able to gain a clear understanding of how commits are
being integrated into the project, along with understanding what changes
these commits are bringing with them. These are the tasks that we set
out to solve with the \mt model. Ideally, we could compare our model
against another that is designed with these goals. To our knowledge, we
know of no git repository visualization tool that has these specific
goals. This may be due to issues with finding the master branch of the
repository which is a non-trivial task, or due to other factors. While
we have found no tool with the express goal of showing how a commit is
integrated, and providing a summarization of a merge, there has been a
lot of work done in providing visualizations of various aspects of a
repository.

Many tools work to address the issues in communication between
developers in inter-team collaboration work. Codebook\cite{Begel2010}
uses a data mining technique to determine the developer of a piece of
code, the program manager who wrote the specification for the code, and
the program managers and developers on the team who were working
together. Hipikat\cite{Cubranic2005} is another tool with a focus on
communication. Where Codebook focuses on developers working on a
project, Hipikat is focused on enabling easier integration of new
developers to a project by providing them with easily-searchable
artifacts of the changes made. Codebook is useful for pairing a
contributor with the original developer; however, the developer may not
have worked with the piece of code in years. A Hipikat program may
provide more information to the maintainer as it records the artifacts
of why certain design decisions were made when they were made using
other tools like Bugzilla and CVS.\@ Neither tool is sufficient in
meeting our goals to provide a summary of the topology of the kernel
repository through a visual tree.

Most visualizers provide a visual presentation of a certain aspect of a
repository. Fractal Figures\cite{Ambros2005} uses a unit square to
represent a portion of a project, then partitions the square based on
the proportion of an author's contributions to that portion of the
project. EPOSee\cite{Burch2005} and Evolution Radar\cite{Ambros2009}
perform further analysis, determine which files are made together, and
what changes are made over a sequence of commits, though the goals
behind these projects is different.

Codebook, Hipikat, Fractal Figures, EPOSee, and Evolution Radar all work
with data from CVS repositories. Our goal is to provide information
about git repositories, specifically the Linux master repository. Fewer
tools are available for generating visualizations and summaries of git
repositories, potentially due to the DAG model used by git.

Gource is a tool for providing an interactive timelapse of the state of
a repository\cite{Caudwell2010}. In the timelapse, it shows who
contributes and what type of contribution a developer is making. These
contribution types are one of, adding a file, removing a file, and
changing a file. While the timelapse is interesting to watch, it does
not provide any additional explanation of the changes actually being
made, only the frequency that they are being made and who is making
them. Codeswarm\cite{ogawa09} is similar to Gource, using an organic approach to
visualizing events in the repository. Unlike Gource, which focuses on
the files, Codeswarm focuses on developers and the number of commits
they are making.

Git itself is shipped with summarization and visualization tools: the
\verb|git log| command and Gitk, which is a graphical tool for the
purpose of browsing the DAG of the repository. Many other tools are
designed with a similar visual metaphore as what is presented in Gitk,
including GitKraken, Github Desktop and web interfaces, the GitLab web
interface, the Bitbucket web interface, are a few examples. The Gitk
interface is built around the central DAG viewer. The DAG displays the
repository events on their respective branches, the author, and the date
that the event was authored. A user can select an event to view addition
information about it, including the full commit log, the parents and
children of the event, and the diff generated if modifications were
made. In the case of the Linux kernel, merges are not resolved as merge
conflicts in the master repository, but by the developers prior to
merging through rebase operations. With this development model, no
merges will have modified any files, and will not show a diff. While the
merge does not make changes directly, the effects of merging the commits
persist, but Gitk is unable to provide an aggregated view of the files
modified, the patches for those files, and the authors who contributed
commits to that merge.

Our tool is primarily aimed at presenting the hierarchical structure of
the Linux git repository. We use tables for presenting the summarized
information of the commits and merges, but this information could also
be presented in a graphical form. Various graphical forms for displaying
file and authorship data exist, the principal forms being matrix views,
city scapes, bar and pie charts, and networks \cite{Eick2002}. Any of
these data visualization metaphors are applicable to our system.

%%% Local Variables:
%%% mode: latex
%%% TeX-master: "lineval.tex"
%%% End:
