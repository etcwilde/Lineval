% vim:set et sw=2 ts=4 tw=72:
\usepackage[T1]{fontenc}
\usepackage[utf8]{inputenc}
\usepackage[hyphens]{url}
\urlstyle{same}

%% Citations
\usepackage[nospace]{cite}

%% Table Support
\usepackage{array}
\usepackage{dcolumn}
\usepackage{longtable}
\usepackage{multirow}
\usepackage{booktabs}
\usepackage{tabulary}

%% Extra support
\usepackage{xspace}
\usepackage{amsmath}
\usepackage{balance}
\usepackage{placeins}

%% Algorithms
\usepackage{algorithm}
\usepackage{algpseudocode}

%% Graphics
\usepackage{tikz}
\usepackage{pgfplots}
\usepackage{pgfplotstable}
\usepackage{xcolor}
\usepackage{color}
\usepackage{listings}


\usetikzlibrary{positioning, arrows}

% Chart Colors
\definecolor{chartblue}{HTML}{3366CC}
\definecolor{chartred}{HTML}{DC3912}
\definecolor{chartyellow}{HTML}{FF9900}
\definecolor{chartgreen}{HTML}{109618}
\definecolor{chartmagenta}{HTML}{990099}
\definecolor{chartpurple}{HTML}{3B3EAC}

% \ifdraft
%     \usepackage[colorinlistoftodos]{todonotes}
%     \newcommand{\evan}[1]{{\color{blue}\emph{Evan Says: #1}}\xspace}
%     \newcommand{\evantodo}[1]{{\color{blue}\emph{Evan Todo: #1}}\xspace}
%     \newcommand{\dmg}[1]{{\color{blue}\emph{dmg Says: #1}}\xspace}
%     \newcommand{\dmgtodo}[1]{{\color{blue}\emph{dmg Todo: #1}}\xspace}
% \else
%     \usepackage[disable]{todonotes}
%     \newcommand{\evan}[1]{}
%     \newcommand{\evantodo}[1]{}
%     \newcommand{\dmg}[1]{}
%     \newcommand{\dmgtodo}[1]{}
% \fi
    \usepackage[colorinlistoftodos]{todonotes}

\newcommand{\tool}{{\emph Linvis}\xspace}


    \newcommand{\evan}[1]{{\color{blue}\emph{Evan Says: #1}}\xspace}
    \newcommand{\evantodo}[1]{{\color{blue}\emph{Evan Todo: #1}}\xspace}
    \newcommand{\dmg}[1]{{\color{blue}\emph{dmg Says: #1}}\xspace}
    \newcommand{\dmgtodo}[1]{{\color{blue}\emph{dmg Todo: #1}}\xspace}


%%% Local Variables:
%%% mode: plain-tex
%%% TeX-master: t
%%% End:


\newcommand{\TheTitle}{Merge-Tree: Visualizing the Integration of Commits into Linux}
\newcommand{\TheAuthors}{Evan Wilde, Daniel M. German}
\newcommand{\TheEmails}{etcwilde@uvic.ca, dmg@uvic.ca}
\newcommand{\TheSubject}{Understanding a large number of commits and merges}
\newcommand{\TheKeywords}{Linux, git, visualizations}

%% Referencing
\usepackage[unicode=true,
          pdftitle={\TheTitle},
          pdfauthor={\TheAuthors},
          colorlinks=false]{hyperref}


\algdef{SE}[DOWHILE]{Do}{doWhile}{\algorithmicdo}[1]{\algorithmicwhile\ #1}%

\lstset{frame=tb,
  language=python,
  aboveskip=3mm,
  belowskip=3mm,
  showstringspaces=false,
  columns=flexible,
  basicstyle={\small\ttfamily},
  numbers=none,
  numberstyle=\tiny\color{gray},
  keywordstyle=\color{chartblue},
  commentstyle=\color{chartred},
  stringstyle=\color{chartgreen},
  breaklines=true,
  breakatwhitespace=true,
  tabsize=3
}

\synctex=1

