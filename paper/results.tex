% vim:set et sw=2 ts=4 tw=72:
% Jun 24, 2017

In this section, we look at the aggreated results from the user study,
and perform some analysis on those results. In the discussion section,
we will further investigate potential reasons behind why various tests
passed or failed.

\subsection{Conceptual Questions Results}

% TODO: make plots showing distribution of times and responses

% Number of commits in the merge tree - Make a plot of this
% | text              | hash      | median |    mean | variance | Actual |
% |-------------------+-----------+--------+---------+----------+--------|
% | Number of commits | a3c1239eb |      4 | 19.1111 | 735.1111 |      1 |
% | Number of commits | cdbdd1676 |      4 |     7.8 | 136.8444 |      5 |
% | Number of merges  | a3c1239eb |      5 |  8.2727 |  53.6182 |      1 |
% | Number of merges  | cdbdd1676 |    3.5 |     5.4 |  50.2667 |      3 |
% Time taken to arrive at these results
% | text              | hash      | median(s) |  mean(s) | variance(s) |
% |-------------------+-----------+-----------+----------+-------------|
% | Number of commits | a3c1239eb |      10.0 |  49.9167 |   5952.0833 |
% | Number of merges  | a3c1239eb |       7.5 |  24.6667 |    884.4242 |
% | Number of commits | cdbdd1676 |      31.5 | 106.8333 |  54123.4242 |
% | Number of merges  | cdbdd1676 |      11.0 |     65.5 |  29798.8182 |

In Table~\ref{tab:conceptual_results}, we are able to see an overview of
the results from the conceptual questions.

\begin{table*}[htpb]
  \centering
  \caption{Results from the conceptual questions}
  \label{tab:conceptual_results}
  \begin{tabular}{ll|r|lrr|rrr}
    Question                      & Commit & Answer & Median & Mean  & Variance & Median(s) & Mean(s) & Variance(s)\\\hline\hline
    Number of commits in the tree & \comA  & 1      & 4      & 19.11 & 753.11   & 10.0      & 49.92   & 5952.08\\
    Number of merges in the tree  & \comA  & 1      & 5      & 8.27  & 53.62    & 7.5       & 24.67   & 884.42\\\hline
    Number of commits in the tree & \comB  & 5      & 4      & 7.80  & 136.84   & 31.5      & 106.83  & 54123.42\\
    Number of merges in the tree  & \comB  & 3      & 3.5    & 5.40  & 50.27    & 11.0      & 65.6    & 29798.82\\
  \end{tabular}
\end{table*}

Users were able to more closely estimate the number of commits and
merges in the larger tree, but generally took longer than the smaller
tree. The tree with a single node resulted in more variability in the
estimate of number of commits .

It should be noted that these questions were answered after spending
roughly ten minutes attesting to draw a picture that held the answers to
these questions.

\subsection{Summarization Question Results}

We will look at the results from each question, and then look at the
combined results to determine the correctness of the answers provided by
the participants.

\subsubsection{What is the series of merges involved with merging this commit}
\label{ssub:what_is_the_series_of_merges_involved_with_merging_this_commit}

	% McNemar's Chi-squared test

% data:  mat
% McNemar's chi-squared = 8, df = 1, p-value = 0.004678

   %         Linvis
% Gitk        Correct Incorrect
  % Correct         2         0
  % Incorrect       8         2


\begin{figure}[htpb]
  \centering
  \includegraphics[width=0.8\linewidth]{figures/userstudy/correct_question-3.pdf}
  \caption{Correctness of participants using \tool versus gitk
    when determining the series of merges involved with merging a given
  commit}
  \label{fig:figures/userstudy/correct_question-3}
\end{figure}

\subsubsection{What other commits are merged in this same merge tree}
\label{ssub:what_other_commits_are_merged_in_this_same_merge_tree}

% McNemar's Chi-squared test

% data:  mat
% McNemar's chi-squared = 7, df = 1, p-value = 0.008151

   %         Linvis
% Gitk        Correct Incorrect
  % Correct         1         0
  % Incorrect       7         4

\begin{figure}[htpb]
  \centering
  \includegraphics[width=0.8\linewidth]{figures/userstudy/correct_question-4.pdf}
  \caption{Correctness of participants using \tool versus gitk when
    determining which other commits are merged into the master branch
    with this commit}
  \label{fig:figures/userstudy/correct_question-4}
\end{figure}

\subsubsection{How many authors are involved with this merge tree}
\label{ssub:how_many_authors_are_involved_with_this_merge_tree}

% McNemar's Chi-squared test

% data:  mat
% McNemar's chi-squared = 9, df = 1, p-value = 0.0027

   %         Linvis
% Gitk        Correct Incorrect
  % Correct         1         0
  % Incorrect       9         2


\begin{figure}[htpb]
  \centering
  \includegraphics[width=0.8\linewidth]{figures/userstudy/correct_question-5.pdf}
  \caption{Correctness of participants using \tool versus gitk when
    determining the number of authors involved in the cluster of commits}
  \label{fig:figures/userstudy/correct_question-5}
\end{figure}

\subsubsection{Who contributed the most changes to this merge tree}
\label{ssub:who_contributed_the_most_changes_to_this_merge_tree}

% McNemar's Chi-squared test

% data:  mat
% McNemar's chi-squared = 8, df = 1, p-value = 0.004678

   %         Linvis
% Gitk        Correct Incorrect
  % Correct         2         0
  % Incorrect       8         2


\begin{figure}[htpb]
  \centering
  \includegraphics[width=0.8\linewidth]{figures/userstudy/correct_question-6.pdf}
  \caption{Correctness of participants using \tool versus gitk when
    determining who contributed the most changes to a merge tree}
  \label{fig:figures/userstudy/correct_question-6}
\end{figure}


\subsubsection{How many files were modified in this merge tree}
\label{ssub:how_many_files_were_modified_in_this_merge_tree}

% McNemar's Chi-squared test

% data:  mat
% McNemar's chi-squared = 9, df = 1, p-value = 0.0027

   %         Linvis
% Gitk        Correct Incorrect
  % Correct         1         0
  % Incorrect       9         2


\begin{figure}[htpb]
  \centering
  \includegraphics[width=0.8\linewidth]{figures/userstudy/correct_question-7.pdf}
  \caption{Correctness of participants using \tool versus gitk when
    determining how many files were modified in a merge tree}
  \label{fig:figures/userstudy/correct_question-7}
\end{figure}

\subsubsection{Which file had the most changes in this merge tree}
\label{ssub:which_file_had_the_most_changes_in_this_merge_tree}

% McNemar's Chi-squared test

% data:  mat
% McNemar's chi-squared = 8, df = 1, p-value = 0.004678

   %         Linvis
% Gitk        Correct Incorrect
  % Correct         1         0
  % Incorrect       8         3


\begin{figure}[htpb]
  \centering
  \includegraphics[width=0.8\linewidth]{figures/userstudy/correct_question-8.pdf}
  \caption{Correctness of participants using \tool versus gitk when
    determining which file(s) had the most changes a merge tree}
  \label{fig:figures/userstudy/correct_question-8}
\end{figure}

\subsubsection{Which modules were involved with this merge tree}
\label{ssub:which_modules_were_involved_with_this_merge_tree}

% McNemar's Chi-squared test

% data:  mat
% McNemar's chi-squared = 1, df = 1, p-value = 0.3173

%             Linvis
% Gitk        Correct Incorrect
  % Correct         8         1
  % Incorrect       3         0

\begin{figure}[htpb]
  \centering
  \includegraphics[width=0.8\linewidth]{figures/userstudy/correct_question-9.pdf}
  \caption{Correctness of participants using \tool versus gitk when
    determining which modules were involved with a merge tree}
  \label{fig:figures/userstudy/correct_question-9}
\end{figure}

\subsubsection{Combined Results}
\label{ssub:combined_results}

% McNemar's Chi-squared test

% data:  mat
% McNemar's chi-squared = 49.075, df = 1, p-value = 2.463e-12


%             Linvis
% Gitk        Correct Incorrect
% Correct     16      1
% Incorrect   52      15
%

For each of the studies, we also tracked the number of people who gave
up on the question, stated that they made a random guess, or replied
with infinity. In other plots, all of these responses are either placed
in the correct or incorrect categories depending on the correctness of
the answer. In Figure~\ref{fig:summarization_correctness}, we show the
combined metrics across all questions, also showing the other
categories.

\begin{figure}[htpb]
  \centering
  \includegraphics[width=0.8\linewidth]{figures/userstudy/summarize_tool_outcomes.pdf}
  \caption{Correctness of participants across all summarization
    questions of participants using \tool versus gitk, including the
    number of instances of giving up on a task, making a random guess,
    or replying with infinity.}
  \label{fig:summarization_correctness}
\end{figure}
