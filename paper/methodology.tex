% vim:set et sw=2 ts=4 tw=72:
% Jun 18, 2017

In order to determine the effectiveness of the merge-tree model on
comprehension and summarization of git repositories, we conducted a user
study on \evantodo{NUMBER} participants. The user study is a
mixed-methods study, but with more focused on the quantitative aspects.
We chose to use a mixed-methods approach to determine both if the model
is able to improve the time performance and accuracy of the
participants, and if the participants found the merge trees more
enjoyable for the given tasks.

In this section, we describe the describe the methods used in the user
study. This includes the participant criteria, the questions asked, the
method for selecting commits, and a breakdown of the methods used for
analyzing the results.

\subsection{User Selection}
\label{sub:user_selection}

Our target audience are git users, particularly users with who are
working on understanding what changes are being made, where they are
being made, and who is making the changes. The merge-tree model is also
intended for users who are trying to learn the structure of a
repository, how modules of the repository are broken into groups, if the

We have software engineering and computer science students at the
Masters, Doctorate, and Post-Doc levels available for participation in
this study. \evan{This feels like the weakest part of the paper, is
  there any way to make it sound stronger?}


\subsection{Commit Selection}
\label{sub:commit_selection}

We chose two commits for use in the study. The order that the commits
are presented is randomized between participants, but the order is kept
consistent through the tasks described in the next subsection.

We use the \tool database, containing commits and merge-tree information
from April 16th 2005 to October 14th, 2014, which corresponds to being
between Linux release 2.6.12-rc3 and Linux 3.17-rc1. \evan{Cite the
  database even more?} We chose the commits based on tree sizes. We
found that a majority of the trees contain at most seven inner commits,
while more than 25\% of the trees contain a single commit. 75\% of the
tree contained up to 51 inner commits and merges, and finally the
largest tree contained 7217 nodes. We chose to work with the trees in
the first and second quartile, as merge trees of sizes between one and
seven, not including the merge into the master branch, make up the
majority of the trees in the database.

From here, we selected one tree from the trees of a single commit at
random. Selecting a commit from that tree is trivial, as there is only
one to choose.

We selected one tree of size seven at random from all trees of size
seven. We placed a restriction the trees, the selected tree must include
at least one internal merge to increase the complexity of the trees
tested. After randomly selecting a tree, we chose in internal commit at
random, using the \verb|random.choice| function in python 3.6.1.

% TODO: Show this in the introduction maybe

% It is worth noting that in the fourth
% quartile, the size of the tree drops off quickly from the top tree, with
% the next tree containing only 4708 nodes, and the third largest tree
% containing only 2349 commits.

Using this technique, we selected commit
\emph{8eb88c80d444fd249edaa7d895666cde79e7b3b8} from the tree containing
a single node (Figure~\ref{fig:commit_1}), and commit
\emph{cdbdd1676a5379f1d5cbd4d476f5e349f445befe}, from the tree
containing seven nodes (Figure~\ref{fig:commit_2}).

\begin{figure}[bpt]
  \centering
  \includegraphics[width=0.08\linewidth]{figures/commits/1-commit.pdf}
  \caption{The first merge tree used in the user study, a merge tree
    containing a single commit}
  \label{fig:commit_1}
\end{figure}

\begin{figure}[bpt]
  \centering
  \includegraphics[width=0.5\linewidth]{figures/commits/7-commits.pdf}
  \caption{The second merge tree used in the user study, a merge tree
    containing seven commits}
  \label{fig:commit_2}
\end{figure}


\subsection{Questions}
\label{sub:questions}

The questions and tasks are broken down into three groups. The first two
groups are purely quantitative in nature, being conceptual tasks, and
summarization tasks. The third group of questions is to determine the
demographic of our participant.

\textbf{Conceptual Tasks}

The participants were allowed to use Gitk or the git command line tools
to perform these tasks. These tasks are asked in this order, allowing
the user to use their diagram to answer the second two questions. It is
not necessary to use \tool in this portion of the study, as it provides
the answers to these questions directly.

\begin{enumerate}
  \item Draw a diagram of the merge tree, showing how the commit was
    merged into the master branch of the repository
  \item How many individual commits are related to this commit?
  \item How many merges are involved with merging this commit into the
    master branch?
\end{enumerate}

We provide the user with 10 minutes per commit to complete the first
task. The answers to the two other questions in this section are drawn
from the answer in the first task.

\textbf{Summarization Tasks}

For these questions, we randomize the order of the tools, choosing to
either start with \tool or Gitk, to ensure that we do not place a bias
on one tool or another through the experiment. The tasks are broken
into task sets related to the overall goal of the task. The task sets
are shuffled, and the tasks within the task sets are shuffled as well.
Finally, the order of the commits is shuffled. We use the
\verb|random.shuffle| function from python 3.6.1 to perform the
shuffling and presentation of the tasks.

\begin{itemize}
  \item Merge task set
    \begin{itemize}
      \item What is the series of merges involved with merging this
        commit?
      \item What other commits are merged in the same merge tree?
    \end{itemize}

  \item Authors task set
    \begin{itemize}
      \item How many authors are involved with this merge tree?
      \item Who contributed the most changes to this merge tree?
    \end{itemize}

  \item Files task set
    \begin{itemize}
      \item How many files were modified in this merge tree?
      \item Which file had the most changes in this merge tree?
    \end{itemize}

  \item Modules task set
    \begin{itemize}
      \item Which modules doe this merge tree involve?
    \end{itemize}
\end{itemize}

As understanding where the merge begins and ends, the merge task set
will always be the first task set presented; however, the questions will
still be re-ordered within the task set.

% TODO: provide the script for question selection
% \lstinputlistings{../selector.py}

\textbf{User Opinion}

Finally, we ask the user for their opinions on the tools and for some
information about their level of experience with git. This portion of
the study is qualitative and does not rely on either of the tool,
instead attempting to determine which model provided the user with a
better experience.

We ask the following questions in this order;

\begin{enumerate}
  \item Given these tasks again, which tool would you prefer to use?
  \item Which aspects of each tool did you like and why?
  \item How long have you used git?
  \item If you have used git, for what kind of projects? (personal,
    school courses, professional?)
  \item If you have used git, how many commits, files, and contributors
    were involved with the largest repository you have worked with?
\end{enumerate}

