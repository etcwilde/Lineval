% vim:set et sw=2 ts=4 tw=72:
% Jul 08, 2017

\section{Background}

Git is a distributed version control system that is used in many
open-source and closed-source projects. It enables many developers to
write code simultaneously, working on separate branches of the same
project. Git uses a directed acyclic graph model (DAG) for storing the
necessary information regarding the state of the code at any point in
time. Git refers to nodes of the DAG as commits, not distinguishing
between the code-carrying commits and the structural commits that bring
two branches together. In this paper, we will refer to the individual
code-carrying commits as commits, the structural merge commits as
merges, and refer to both types without discrimination as repository
events.

Each repository event contains an ordered list of one or more parents,
except the initial commit which has no parents. The parent relationship
points from the current event toward the initial commit.  Commits have a
single parent, while merges will have at least two parents, the parent
on the current branch followed by the branches being merged into the
current branch. Git allows for the merging of multiple branches
simultaneously in octopus merges, where the order of the parents
represents the order that the branches were specified for being merged\footnote{One octopus merge in Linux
  (2cde51fbd0f310c8a2c5f977e665c0ac3945b46d) has 66 parents.}.
A \foxtrot\footnote{See
  \url{http://bit-booster.blogspot.ca/2016/02/no-foxtrots-allowed.html}
  for a full description of the issue.} merge occurs when the parent on
the current branch is replaced by a parent from a branch being merged
into the current branch, thus confounding the branch relationships.

% TODO: Get a diagram of a foxtrot

The repository events in the DAG are immutable; once a repository event
is created, it can't be changed. Git allows operations to alter
repository events and re-order them, but this will create a new event
with a new commit hash. This property makes a repository event unable to
record the traversal of merges into the master branch of the repository.
Introducing an additional field to track the path to the master branch
would create a circular dependency between the parent and child. Any
updates to the DAG would modify the commit hashes for all repository
events in the repository. To alleviate this, git provides the command
\verb|git log --children| which traverses the DAG and prints the
inverted edges relationship between events.

Under perfect conditions, visualizations of the DAG show how a
repository changes over time; where logical branches are being made,
where merges are happening, and who is making the changes. Most
repositories are relatively simple, with relatively few, small, branches
with a clear, well defined master branch. However, in large, active
repositories it becomes very difficult for developers to use the DAG to
identify the master branch, and as a consequence, where branches start
and are merged on the path to the master branch. \evan{They do show in
  perfect conditions. If you don't have many branches, and you don't
  have any foxtrots, the first branch on the left of the DAG
  visualization WILL be the master branch}

%%% Local Variables:
%%% mode: latex
%%% TeX-master: "lineval.tex"
%%% End:
