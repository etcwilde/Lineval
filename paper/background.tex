% vim:set et sw=2 ts=4 tw=72:
% Jul 08, 2017

Git is a distributed version control system that is used in many
open-source and closed-source projects. It enables many developers to
write code simultaneously, working on separate branches of the same
project. Git uses a directed acyclic graph model (DAG) for storing the
necessary information regarding the state of the code at any point in
time. Git refers to nodes of the DAG as commits, not distinguishing
between the code-carrying commits and the structural commits that bring
two branches together. In this paper, we will refer to the individual
commits code-carrying as commits, the structural merge commits as
merges, and refer to both types without discrimination as repository
events.

Each repository event contains an ordered list of one or more parents,
except the initial commit which has no parents. The parent relationship
points from the current event toward the initial commit.  Commits have a
single parent, while merges will have at least two parents, the parent
on the current branch followed by the branches being merged into the
current branch. Git allows for the merging of multiple branches
simultaneously in octopus merges, where the order of the parents
represents the order that the branches were specified for being merged.
A \foxtrot\footnote{See
  \url{http://bit-booster.blogspot.ca/2016/02/no-foxtrots-allowed.html}
  for a full description of the issue.} merge occurs when the parent on
the current branch is replaced by a parent from a branch being merged
into the current branch, thus confounding the branch relationships.

The repository events in the DAG are immutable; once a repository event
is created, it can't be changed. Git allows operations to alter commits
and re-order them, but this will create a new commit with a new commit
hash.  This property makes a repository event unable to record the
traversal of merges into the master branch of the repository. A commit
cannot have references in the direction of the merge that merges the
commit, as this reference will change, requiring the commit to be
changed. While this provides some challenge, git eliminates this issue
of the DAG with the \verb|git log --children| command, which traverses
the DAG and inverts the edges to provide the reference from parent to
child.

Under perfect conditions, visualizations of the DAG show how a
repository changes over time; where logical branches are being made,
where merges are happening, and who is making the changes. Most
repositories do not portray the perfect conditions, where there are
relatively few branches and the merges are well-behaved, leaving a
well-defined master branch. Most repositories do not mantain the perfect
structure; either introducing \foxtrot merges, containing many commits,
or many branches, making the DAG difficult to interpret.
