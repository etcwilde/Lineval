Merge-trees, as presented by Wilde and German\cite{Wilde2016}, are a
data structure designed to make the navigation and summarization of git
merges simpler than using the directed acyclic graph (DAG), as done in
gitk among other tools\evan{does this need a citation?}. Git uses the
DAG model internally, making it simple to construct visualization tools
directly using the DAG of the repository;\@ however, this model does not
provide a clear summarization of the events occuring in the repository.
Gitk and the git command line interface (cli) both provide the user with
a visualization of the DAG of the repository.

Merge-trees have two primary issues; first, the merge-tree model has not
been evaulated, and cannot therefor claim to simplify the summarization
and navigation tasks. Second, merge-trees have only been implemented for
the Linux repository, which follows a strict merge-policy where only
Linus can merge into the master branch.

The contributions of this paper tackle both of these issues. We perform
a user-study, to verify that the merge-tree model is able to improve the
accuracy and speed of summarization tasks. Second, we generalize the
algorithm, creating an upper bound on the number of commits visisted in
the construction of the merge-trees.

\subsection{Merge-Trees}
\label{sub:merge_trees}

Internally, git uses a DAG as the primary data structure for managing
commits in a repository. An individual commit will have an ordered list
of parents, where the parent in the first position is on the same branch
as the commit, and the other parents are in the same order as they were
merged. In the event that a merge commit contains a list of parents with
more than two parents, this merge is an octopus merge. The commits may
be divided into two groups, commits with only a single parent, and
commits with at least two commits. Commits with with many parents are
merge commits, combining the effects of potentially many individual
commits.
